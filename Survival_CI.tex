\documentclass[]{article}
\usepackage{lmodern}
\usepackage{amssymb,amsmath}
\usepackage{ifxetex,ifluatex}
\usepackage{fixltx2e} % provides \textsubscript
\ifnum 0\ifxetex 1\fi\ifluatex 1\fi=0 % if pdftex
  \usepackage[T1]{fontenc}
  \usepackage[utf8]{inputenc}
\else % if luatex or xelatex
  \ifxetex
    \usepackage{mathspec}
  \else
    \usepackage{fontspec}
  \fi
  \defaultfontfeatures{Ligatures=TeX,Scale=MatchLowercase}
\fi
% use upquote if available, for straight quotes in verbatim environments
\IfFileExists{upquote.sty}{\usepackage{upquote}}{}
% use microtype if available
\IfFileExists{microtype.sty}{%
\usepackage{microtype}
\UseMicrotypeSet[protrusion]{basicmath} % disable protrusion for tt fonts
}{}
\usepackage[margin=1in]{geometry}
\usepackage{hyperref}
\hypersetup{unicode=true,
            pdftitle={Survival simulation},
            pdfauthor={Uni:jr3755},
            pdfborder={0 0 0},
            breaklinks=true}
\urlstyle{same}  % don't use monospace font for urls
\usepackage{color}
\usepackage{fancyvrb}
\newcommand{\VerbBar}{|}
\newcommand{\VERB}{\Verb[commandchars=\\\{\}]}
\DefineVerbatimEnvironment{Highlighting}{Verbatim}{commandchars=\\\{\}}
% Add ',fontsize=\small' for more characters per line
\usepackage{framed}
\definecolor{shadecolor}{RGB}{248,248,248}
\newenvironment{Shaded}{\begin{snugshade}}{\end{snugshade}}
\newcommand{\KeywordTok}[1]{\textcolor[rgb]{0.13,0.29,0.53}{\textbf{#1}}}
\newcommand{\DataTypeTok}[1]{\textcolor[rgb]{0.13,0.29,0.53}{#1}}
\newcommand{\DecValTok}[1]{\textcolor[rgb]{0.00,0.00,0.81}{#1}}
\newcommand{\BaseNTok}[1]{\textcolor[rgb]{0.00,0.00,0.81}{#1}}
\newcommand{\FloatTok}[1]{\textcolor[rgb]{0.00,0.00,0.81}{#1}}
\newcommand{\ConstantTok}[1]{\textcolor[rgb]{0.00,0.00,0.00}{#1}}
\newcommand{\CharTok}[1]{\textcolor[rgb]{0.31,0.60,0.02}{#1}}
\newcommand{\SpecialCharTok}[1]{\textcolor[rgb]{0.00,0.00,0.00}{#1}}
\newcommand{\StringTok}[1]{\textcolor[rgb]{0.31,0.60,0.02}{#1}}
\newcommand{\VerbatimStringTok}[1]{\textcolor[rgb]{0.31,0.60,0.02}{#1}}
\newcommand{\SpecialStringTok}[1]{\textcolor[rgb]{0.31,0.60,0.02}{#1}}
\newcommand{\ImportTok}[1]{#1}
\newcommand{\CommentTok}[1]{\textcolor[rgb]{0.56,0.35,0.01}{\textit{#1}}}
\newcommand{\DocumentationTok}[1]{\textcolor[rgb]{0.56,0.35,0.01}{\textbf{\textit{#1}}}}
\newcommand{\AnnotationTok}[1]{\textcolor[rgb]{0.56,0.35,0.01}{\textbf{\textit{#1}}}}
\newcommand{\CommentVarTok}[1]{\textcolor[rgb]{0.56,0.35,0.01}{\textbf{\textit{#1}}}}
\newcommand{\OtherTok}[1]{\textcolor[rgb]{0.56,0.35,0.01}{#1}}
\newcommand{\FunctionTok}[1]{\textcolor[rgb]{0.00,0.00,0.00}{#1}}
\newcommand{\VariableTok}[1]{\textcolor[rgb]{0.00,0.00,0.00}{#1}}
\newcommand{\ControlFlowTok}[1]{\textcolor[rgb]{0.13,0.29,0.53}{\textbf{#1}}}
\newcommand{\OperatorTok}[1]{\textcolor[rgb]{0.81,0.36,0.00}{\textbf{#1}}}
\newcommand{\BuiltInTok}[1]{#1}
\newcommand{\ExtensionTok}[1]{#1}
\newcommand{\PreprocessorTok}[1]{\textcolor[rgb]{0.56,0.35,0.01}{\textit{#1}}}
\newcommand{\AttributeTok}[1]{\textcolor[rgb]{0.77,0.63,0.00}{#1}}
\newcommand{\RegionMarkerTok}[1]{#1}
\newcommand{\InformationTok}[1]{\textcolor[rgb]{0.56,0.35,0.01}{\textbf{\textit{#1}}}}
\newcommand{\WarningTok}[1]{\textcolor[rgb]{0.56,0.35,0.01}{\textbf{\textit{#1}}}}
\newcommand{\AlertTok}[1]{\textcolor[rgb]{0.94,0.16,0.16}{#1}}
\newcommand{\ErrorTok}[1]{\textcolor[rgb]{0.64,0.00,0.00}{\textbf{#1}}}
\newcommand{\NormalTok}[1]{#1}
\usepackage{graphicx,grffile}
\makeatletter
\def\maxwidth{\ifdim\Gin@nat@width>\linewidth\linewidth\else\Gin@nat@width\fi}
\def\maxheight{\ifdim\Gin@nat@height>\textheight\textheight\else\Gin@nat@height\fi}
\makeatother
% Scale images if necessary, so that they will not overflow the page
% margins by default, and it is still possible to overwrite the defaults
% using explicit options in \includegraphics[width, height, ...]{}
\setkeys{Gin}{width=\maxwidth,height=\maxheight,keepaspectratio}
\IfFileExists{parskip.sty}{%
\usepackage{parskip}
}{% else
\setlength{\parindent}{0pt}
\setlength{\parskip}{6pt plus 2pt minus 1pt}
}
\setlength{\emergencystretch}{3em}  % prevent overfull lines
\providecommand{\tightlist}{%
  \setlength{\itemsep}{0pt}\setlength{\parskip}{0pt}}
\setcounter{secnumdepth}{0}
% Redefines (sub)paragraphs to behave more like sections
\ifx\paragraph\undefined\else
\let\oldparagraph\paragraph
\renewcommand{\paragraph}[1]{\oldparagraph{#1}\mbox{}}
\fi
\ifx\subparagraph\undefined\else
\let\oldsubparagraph\subparagraph
\renewcommand{\subparagraph}[1]{\oldsubparagraph{#1}\mbox{}}
\fi

%%% Use protect on footnotes to avoid problems with footnotes in titles
\let\rmarkdownfootnote\footnote%
\def\footnote{\protect\rmarkdownfootnote}

%%% Change title format to be more compact
\usepackage{titling}

% Create subtitle command for use in maketitle
\newcommand{\subtitle}[1]{
  \posttitle{
    \begin{center}\large#1\end{center}
    }
}

\setlength{\droptitle}{-2em}

  \title{Survival simulation}
    \pretitle{\vspace{\droptitle}\centering\huge}
  \posttitle{\par}
    \author{Uni:jr3755}
    \preauthor{\centering\large\emph}
  \postauthor{\par}
      \predate{\centering\large\emph}
  \postdate{\par}
    \date{February 6, 2019}


\begin{document}
\maketitle

\section{Simulation function for exponential or
Weibull}\label{simulation-function-for-exponential-or-weibull}

\begin{Shaded}
\begin{Highlighting}[]
\CommentTok{# baseline hazard: Weibull}

\CommentTok{# N = sample size    }
\CommentTok{# lambda = scale parameter in h0()}
\CommentTok{# rho = shape parameter in h0()}
\CommentTok{# beta = fixed effect parameter}

\NormalTok{simu_data <-}\StringTok{ }\ControlFlowTok{function}\NormalTok{(N, lambda, rho, beta, }\DataTypeTok{baseline =} \StringTok{"Exponential"}\NormalTok{)}
\NormalTok{\{}
  \CommentTok{# covariate --> N Bernoulli trials}
\NormalTok{  x <-}\StringTok{ }\KeywordTok{sample}\NormalTok{(}\DataTypeTok{x=}\KeywordTok{c}\NormalTok{(}\DecValTok{0}\NormalTok{, }\DecValTok{1}\NormalTok{), }\DataTypeTok{size=}\NormalTok{N, }\DataTypeTok{replace=}\OtherTok{TRUE}\NormalTok{, }\DataTypeTok{prob=}\KeywordTok{c}\NormalTok{(}\FloatTok{0.5}\NormalTok{, }\FloatTok{0.5}\NormalTok{))}
\NormalTok{  v <-}\StringTok{ }\KeywordTok{runif}\NormalTok{(}\DataTypeTok{n=}\NormalTok{N)}
  \ControlFlowTok{if}\NormalTok{ (baseline }\OperatorTok{==}\StringTok{ "Exponential"}\NormalTok{) \{}
\NormalTok{    Tlat <-}\StringTok{ }\KeywordTok{log}\NormalTok{(v)}\OperatorTok{/}\NormalTok{(}\OperatorTok{-}\KeywordTok{exp}\NormalTok{(x }\OperatorTok{*}\StringTok{ }\NormalTok{beta) }\OperatorTok{*}\StringTok{ }\NormalTok{lambda)\}}
  \ControlFlowTok{else}\NormalTok{\{}
    \CommentTok{# Weibull latent event times}
\NormalTok{    Tlat <-}\StringTok{ }\NormalTok{(}\OperatorTok{-}\StringTok{ }\KeywordTok{log}\NormalTok{(v) }\OperatorTok{/}\StringTok{ }\NormalTok{(lambda }\OperatorTok{*}\StringTok{ }\KeywordTok{exp}\NormalTok{(x }\OperatorTok{*}\StringTok{ }\NormalTok{beta)))}\OperatorTok{^}\NormalTok{(}\DecValTok{1} \OperatorTok{/}\StringTok{ }\NormalTok{rho)}
\NormalTok{  \}}

  \CommentTok{# data set}
  \KeywordTok{data.frame}\NormalTok{(}\DataTypeTok{id=}\DecValTok{1}\OperatorTok{:}\NormalTok{N,}
             \DataTypeTok{time=}\NormalTok{Tlat,}
             \DataTypeTok{x=}\NormalTok{x,}
             \DataTypeTok{baseline =}\NormalTok{ baseline, }\DataTypeTok{stringsAsFactors =} \OtherTok{FALSE}\NormalTok{, }\DataTypeTok{row.names =} \OtherTok{NULL}\NormalTok{)}
\NormalTok{\}}
\end{Highlighting}
\end{Shaded}

\begin{Shaded}
\begin{Highlighting}[]
\CommentTok{# #Simulation}
\CommentTok{# set.seed(1234)}
\CommentTok{# sim_num = 100 #Number of simulations}
\CommentTok{# betaHat <- rep(NA, sim_num) #Initial mean beta for each simulation}
\CommentTok{# baseline = "Cox" #The model for generated data}
\CommentTok{# fitmodel = "Cox" #The model to fit data}
\CommentTok{# for(k in 1:sim_num)}
\CommentTok{# \{}
\CommentTok{#   dat <- simu_data(N=100, lambda=0.01, rho=1, beta=-0.6, baseline = baseline)}
\CommentTok{#   if(fitmodel == "Exponential")\{}
\CommentTok{#     fit <- survreg(Surv(time) ~ x, data = dat, dist = "exponential")}
\CommentTok{#     betaHat[k] <- -fit$coefficients[-1]}
\CommentTok{#   \} else if(fitmodel == "Cox")\{}
\CommentTok{#     fit <- coxph(Surv(time) ~ x, data=dat)}
\CommentTok{#     betaHat[k] <- fit$coefficients[-1]}
\CommentTok{#   \}else\{}
\CommentTok{#     fit <- survreg(Surv(time) ~ x, data = dat, dist = "weibull")}
\CommentTok{#     betaHat[k] <- -fit$coefficients[-1] / fit$scale}
\CommentTok{#   \}}
\CommentTok{# \}}
\CommentTok{# betaHat #The returned list of mean beta for each simulation}
\end{Highlighting}
\end{Shaded}

\section{Xinyao Wu}\label{xinyao-wu}

\section{merge Simulation into a
function}\label{merge-simulation-into-a-function}

\begin{Shaded}
\begin{Highlighting}[]
\KeywordTok{set.seed}\NormalTok{(}\DecValTok{1234}\NormalTok{)}
\NormalTok{sim =}\StringTok{ }\ControlFlowTok{function}\NormalTok{(fitmodel,baseline,sim_num)\{}
\NormalTok{  num =}\StringTok{ }\NormalTok{sim_num}
\NormalTok{  betaHat =}\StringTok{ }\KeywordTok{rep}\NormalTok{(}\OtherTok{NA}\NormalTok{, num) }
  \ControlFlowTok{for}\NormalTok{(k }\ControlFlowTok{in} \DecValTok{1}\OperatorTok{:}\NormalTok{sim_num)}
\NormalTok{\{}
\NormalTok{  dat <-}\StringTok{ }\KeywordTok{simu_data}\NormalTok{(}\DataTypeTok{N=}\DecValTok{100}\NormalTok{, }\DataTypeTok{lambda=}\FloatTok{0.01}\NormalTok{, }\DataTypeTok{rho=}\DecValTok{1}\NormalTok{, }\DataTypeTok{beta=}\OperatorTok{-}\FloatTok{0.6}\NormalTok{, }\DataTypeTok{baseline =}\NormalTok{ baseline)}
  \ControlFlowTok{if}\NormalTok{(fitmodel }\OperatorTok{==}\StringTok{ "Exponential"}\NormalTok{)\{}
\NormalTok{    fit <-}\StringTok{ }\KeywordTok{survreg}\NormalTok{(}\KeywordTok{Surv}\NormalTok{(time) }\OperatorTok{~}\StringTok{ }\NormalTok{x, }\DataTypeTok{data =}\NormalTok{ dat, }\DataTypeTok{dist =} \StringTok{"exponential"}\NormalTok{)}
\NormalTok{    betaHat[k] <-}\StringTok{ }\OperatorTok{-}\NormalTok{fit}\OperatorTok{$}\NormalTok{coefficients[}\OperatorTok{-}\DecValTok{1}\NormalTok{]}
\NormalTok{  \} }\ControlFlowTok{else} \ControlFlowTok{if}\NormalTok{(fitmodel }\OperatorTok{==}\StringTok{ "Cox"}\NormalTok{)\{}
\NormalTok{    fit <-}\StringTok{ }\KeywordTok{coxph}\NormalTok{(}\KeywordTok{Surv}\NormalTok{(time) }\OperatorTok{~}\StringTok{ }\NormalTok{x, }\DataTypeTok{data=}\NormalTok{dat)}
\NormalTok{    betaHat[k] <-}\StringTok{ }\NormalTok{fit}\OperatorTok{$}\NormalTok{coefficients[}\DecValTok{1}\NormalTok{]}
\NormalTok{  \}}\ControlFlowTok{else}\NormalTok{\{}
\NormalTok{    fit <-}\StringTok{ }\KeywordTok{survreg}\NormalTok{(}\KeywordTok{Surv}\NormalTok{(time) }\OperatorTok{~}\StringTok{ }\NormalTok{x, }\DataTypeTok{data =}\NormalTok{ dat, }\DataTypeTok{dist =} \StringTok{"weibull"}\NormalTok{)}
\NormalTok{    betaHat[k] <-}\StringTok{ }\OperatorTok{-}\NormalTok{fit}\OperatorTok{$}\NormalTok{coefficients[}\OperatorTok{-}\DecValTok{1}\NormalTok{] }\OperatorTok{/}\StringTok{ }\NormalTok{fit}\OperatorTok{$}\NormalTok{scale}
\NormalTok{  \}}
\NormalTok{  \}}
\NormalTok{   betaHat}
\NormalTok{  \}}
\end{Highlighting}
\end{Shaded}

\begin{Shaded}
\begin{Highlighting}[]
\CommentTok{# #try}
\CommentTok{# #Exponential}
\CommentTok{# beta_exp_100 = sim("Weibull","Exponential",100)}
\CommentTok{# hist(beta_exp_100)}
\CommentTok{# mean(beta_exp_100)}
\CommentTok{# ci(beta_exp_100,confidence = 0.95)}
\end{Highlighting}
\end{Shaded}

\begin{Shaded}
\begin{Highlighting}[]
\KeywordTok{library}\NormalTok{(gmodels)}
\NormalTok{model_list =}\StringTok{ }\KeywordTok{list}\NormalTok{(}\StringTok{"Exponential"}\NormalTok{,}\StringTok{"Weibull"}\NormalTok{,}\StringTok{"Cox"}\NormalTok{)}
\NormalTok{beta_hat =}\StringTok{ }\KeywordTok{vector}\NormalTok{(}\StringTok{"list"}\NormalTok{, }\DataTypeTok{length =} \DecValTok{9}\NormalTok{)}
\NormalTok{result =}\StringTok{ }\KeywordTok{tibble}\NormalTok{()}
\NormalTok{k =}\DecValTok{1}
\ControlFlowTok{for}\NormalTok{(i }\ControlFlowTok{in} \DecValTok{1}\OperatorTok{:}\DecValTok{3}\NormalTok{)\{}
  \ControlFlowTok{for}\NormalTok{(j }\ControlFlowTok{in} \DecValTok{1}\OperatorTok{:}\DecValTok{3}\NormalTok{)\{}
\NormalTok{  beta_hat[[k]]=}\StringTok{ }\KeywordTok{sim}\NormalTok{(model_list[[i]],model_list[[j]],}\DecValTok{100}\NormalTok{)}
\NormalTok{  result[k,}\DecValTok{1}\NormalTok{] =}\StringTok{ }\NormalTok{model_list[[i]]}
\NormalTok{  result[k,}\DecValTok{2}\NormalTok{] =}\StringTok{ }\NormalTok{model_list[[j]]}
\NormalTok{  result[k,}\DecValTok{3}\NormalTok{] =}\StringTok{ }\KeywordTok{ci}\NormalTok{(beta_hat[[k]])[[}\DecValTok{1}\NormalTok{]]}
\NormalTok{  result[k,}\DecValTok{4}\NormalTok{] =}\StringTok{ }\KeywordTok{ci}\NormalTok{(beta_hat[[k]])[[}\DecValTok{2}\NormalTok{]]}
\NormalTok{  result[k,}\DecValTok{5}\NormalTok{] =}\StringTok{ }\KeywordTok{ci}\NormalTok{(beta_hat[[k]])[[}\DecValTok{3}\NormalTok{]]}
\NormalTok{  result[k,}\DecValTok{6}\NormalTok{] =}\StringTok{ }\KeywordTok{ci}\NormalTok{(beta_hat[[k]])[[}\DecValTok{4}\NormalTok{]]}
\NormalTok{  k =}\StringTok{ }\NormalTok{k}\OperatorTok{+}\DecValTok{1}
\NormalTok{  \}}
\NormalTok{\}}
\end{Highlighting}
\end{Shaded}

\begin{verbatim}
## Warning in ci.numeric(beta_hat[[k]]): No class or unkown class. Using
## default calcuation.

## Warning in ci.numeric(beta_hat[[k]]): No class or unkown class. Using
## default calcuation.

## Warning in ci.numeric(beta_hat[[k]]): No class or unkown class. Using
## default calcuation.

## Warning in ci.numeric(beta_hat[[k]]): No class or unkown class. Using
## default calcuation.

## Warning in ci.numeric(beta_hat[[k]]): No class or unkown class. Using
## default calcuation.

## Warning in ci.numeric(beta_hat[[k]]): No class or unkown class. Using
## default calcuation.

## Warning in ci.numeric(beta_hat[[k]]): No class or unkown class. Using
## default calcuation.

## Warning in ci.numeric(beta_hat[[k]]): No class or unkown class. Using
## default calcuation.

## Warning in ci.numeric(beta_hat[[k]]): No class or unkown class. Using
## default calcuation.

## Warning in ci.numeric(beta_hat[[k]]): No class or unkown class. Using
## default calcuation.

## Warning in ci.numeric(beta_hat[[k]]): No class or unkown class. Using
## default calcuation.

## Warning in ci.numeric(beta_hat[[k]]): No class or unkown class. Using
## default calcuation.

## Warning in ci.numeric(beta_hat[[k]]): No class or unkown class. Using
## default calcuation.

## Warning in ci.numeric(beta_hat[[k]]): No class or unkown class. Using
## default calcuation.

## Warning in ci.numeric(beta_hat[[k]]): No class or unkown class. Using
## default calcuation.

## Warning in ci.numeric(beta_hat[[k]]): No class or unkown class. Using
## default calcuation.

## Warning in ci.numeric(beta_hat[[k]]): No class or unkown class. Using
## default calcuation.

## Warning in ci.numeric(beta_hat[[k]]): No class or unkown class. Using
## default calcuation.

## Warning in ci.numeric(beta_hat[[k]]): No class or unkown class. Using
## default calcuation.

## Warning in ci.numeric(beta_hat[[k]]): No class or unkown class. Using
## default calcuation.

## Warning in ci.numeric(beta_hat[[k]]): No class or unkown class. Using
## default calcuation.

## Warning in ci.numeric(beta_hat[[k]]): No class or unkown class. Using
## default calcuation.

## Warning in ci.numeric(beta_hat[[k]]): No class or unkown class. Using
## default calcuation.

## Warning in ci.numeric(beta_hat[[k]]): No class or unkown class. Using
## default calcuation.

## Warning in ci.numeric(beta_hat[[k]]): No class or unkown class. Using
## default calcuation.

## Warning in ci.numeric(beta_hat[[k]]): No class or unkown class. Using
## default calcuation.

## Warning in ci.numeric(beta_hat[[k]]): No class or unkown class. Using
## default calcuation.

## Warning in ci.numeric(beta_hat[[k]]): No class or unkown class. Using
## default calcuation.

## Warning in ci.numeric(beta_hat[[k]]): No class or unkown class. Using
## default calcuation.

## Warning in ci.numeric(beta_hat[[k]]): No class or unkown class. Using
## default calcuation.

## Warning in ci.numeric(beta_hat[[k]]): No class or unkown class. Using
## default calcuation.

## Warning in ci.numeric(beta_hat[[k]]): No class or unkown class. Using
## default calcuation.

## Warning in ci.numeric(beta_hat[[k]]): No class or unkown class. Using
## default calcuation.

## Warning in ci.numeric(beta_hat[[k]]): No class or unkown class. Using
## default calcuation.

## Warning in ci.numeric(beta_hat[[k]]): No class or unkown class. Using
## default calcuation.

## Warning in ci.numeric(beta_hat[[k]]): No class or unkown class. Using
## default calcuation.
\end{verbatim}

\begin{Shaded}
\begin{Highlighting}[]
\KeywordTok{names}\NormalTok{(result)=}\KeywordTok{c}\NormalTok{(}\StringTok{"base_model"}\NormalTok{,}\StringTok{"fit_model"}\NormalTok{,}\StringTok{"beta_Estimate"}\NormalTok{,}\StringTok{"CI lower"}\NormalTok{,}\StringTok{"CI upper"}\NormalTok{,}\StringTok{"Std. Error"}\NormalTok{)}
\NormalTok{result}
\end{Highlighting}
\end{Shaded}

\begin{verbatim}
## # A tibble: 9 x 6
##   base_model  fit_model   beta_Estimate `CI lower` `CI upper` `Std. Error`
## * <chr>       <chr>               <dbl>      <dbl>      <dbl>        <dbl>
## 1 Exponential Exponential        -0.583     -0.625     -0.541       0.0210
## 2 Exponential Weibull            -0.580     -0.620     -0.541       0.0198
## 3 Exponential Cox                -0.606     -0.646     -0.566       0.0202
## 4 Weibull     Exponential        -0.609     -0.655     -0.563       0.0233
## 5 Weibull     Weibull            -0.650     -0.698     -0.602       0.0242
## 6 Weibull     Cox                -0.610     -0.650     -0.570       0.0202
## 7 Cox         Exponential        -0.628     -0.674     -0.582       0.0232
## 8 Cox         Weibull            -0.602     -0.644     -0.560       0.0212
## 9 Cox         Cox                -0.658     -0.701     -0.615       0.0217
\end{verbatim}


\end{document}
